% based on example 7 in pythontex_gallery
% https://github.com/gpoore/pythontex/

\documentclass[12pt]{mpllatex}
\usepackage{examples}

\begin{document}

\section*{A table of derivatives and anti-derivatives}

This example is based upon a nice example in the Pythontex gallery, see
\ \url{https://github.com/gpoore/pythontex/}.
It uses a tagged block to capture the Maple output for later use
in the body of the LaTeX table.

\lstset{numbers=left}

\begin{minipage}[t]{0.75\textwidth}
\begin{maple}
   # Create a list of functions to include in the table
   funcs := [[sin(x),"\\\\"],         [cos(x),"\\\\"],         [tan(x),"\\\\"],
             [arcsin(x),"\\\\[5pt]"], [arccos(x),"\\\\[5pt]"], [arctan(x),"\\\\[5pt]"],
             [sinh(x),"\\\\"],        [cosh(x),"\\\\"],        [tanh(x)," "]]:

   # mplBeg (CalculusTable)
   for foo in funcs do
       func := foo[1]:
       eol  := foo[2]:
       myddx := ''diff''(func,x):
       myint := ''int''(func,x):
       Print(cat(Latex(myddx),"&=",Latex(diff(func,x)),"\\quad & \\quad")):
       Print(cat(Latex(myint),"&=",Latex(int(func,x)),eol)):
   end do:
   # mplEnd (CalculusTable)
\end{maple}
\end{minipage}
\hskip 1cm
\begin{minipage}[t]{0.25\textwidth}
\begin{latex}
   \begin{align*}
      \mpl {CalculusTable}
   \end{align*}
\end{latex}
\end{minipage}

\clearpage

\begin{align*}
   \mpl {CalculusTable}
\end{align*}

\end{document}
