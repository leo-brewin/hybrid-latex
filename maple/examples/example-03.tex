% based on example 8 in pythontex_gallery
% https://github.com/gpoore/pythontex/

\documentclass[12pt]{mpllatex}
\usepackage{examples}

\begin{document}

\section*{Step-by-step integration}

This is another nice example drawn from the Pythontex gallery, see
\ \url{https://github.com/gpoore/pythontex}.

It shows the step-by-step computations of a simple triple integral.

\vspace{12pt}

\begin{maple}
   # Define limits of integration
   x_max := 2:   y_max := 3:   z_max := 4:
   x_min := 0:   y_min := 0:   z_min := 0:

   ans := int(f(x,y,z), [x=x_min..x_max, y=y_min..y_max, z=z_min..z_max]):            # mpl(lhs.01,ans)

   f := (x,y,z) -> x*y + y*sin(z) + cos(x+y):

   ans := ''int''(''int''(''int''(f(x,y,z), x=x_min..x_max), y=y_min..y_max), z=z_min..z_max):    # mpl(rhs.01,ans)
   ans := ''int''(''int''(int(f(x,y,z), x=x_min..x_max), y=y_min..y_max), z=z_min..z_max):        # mpl(rhs.02,ans)
   ans := ''int''(int(int(f(x,y,z), x=x_min..x_max), y=y_min..y_max), z=z_min..z_max):            # mpl(rhs.03,ans)
   ans := int(int(int(f(x,y,z), x=x_min..x_max), y=y_min..y_max), z=z_min..z_max):                # mpl(rhs.04,ans)

   # And now, a numerical approximation
   ans := evalf[15](ans):                                                             # mpl(rhs.05,ans)

\end{maple}

\begin{minipage}[t]{0.65\textwidth}
\begin{align*}
   \mpl{lhs.01} &= \mpl{rhs.01}\\
                &= \mpl{rhs.02}\\
                &= \mpl{rhs.03}\\
                &= \mpl{rhs.04}\\
                &\approx \mpl{rhs.05}
\end{align*}
\end{minipage}
\hskip 1cm
\lower16pt\hbox{%
\begin{minipage}[t]{0.35\textwidth}
\begin{latex}
   \begin{align*}
      \mpl{lhs.01} &= \mpl{rhs.01}\\
                   &= \mpl{rhs.02}\\
                   &= \mpl{rhs.03}\\
                   &= \mpl{rhs.04}\\
                   &\approx \mpl{rhs.05}
   \end{align*}
\end{latex}
\end{minipage}}

\end{document}
