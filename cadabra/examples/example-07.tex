\documentclass[12pt]{cdblatex}
\usepackage{examples}

\lstset{numbers=left}

\begin{document}

\section*{The metric connection in Riemann normal coordinates}

In local Riemann normal coordinates, the metric components can always be expanded as a power series in the Riemann curvatures and its derivatives (provided the curvatives are finite at the expansion point). In particular
\begin{align*}
   g_{ab}(x) &= g_{a b} - \frac{1}{3} x^{c} x^{d} R_{a c b d}
                        - \frac{1}{6} x^{c} x^{d} x^{e} \nabla_{c}{R_{a d b e}} + \cdots\\
   g^{ab}(x) &= g^{a b} + \frac{1}{3} x^{c} x^{d} g^{a e} g^{b f} R_{c e d f}
                        + \frac{1}{6} x^{c} x^{d} x^{e} g^{a f} g^{b g} \nabla_{c}{R_{d f e g}} + \cdots
\end{align*}
where $g_{ab}$ and $g^{ab}$ are independent of the coordinates $x^a$ and where $\nabla$ is the metric compatable derivative operator (i.e., $\nabla(g)=0$). In applications in General Relativity the $g_{ab}$ are often chosen to be $g_{ab} = {\rm diag}(-1,1,1,1)$.

Here we will use the standard metric compatible connection
\begin{align*}
   \Gamma^{d}_{ab}(x) = \frac{1}{2} g^{dc}\left( g_{cb,a} + g_{ac,b} - g_{ab,c} \right)
\end{align*}
to compute $\Gamma^{d}_{ab}(x)$ to terms linear in $R_{abcd}$ and $\nabla_{e} R_{abcd}$.

\vspace{12pt}

\begin{cadabra}
   {a,b,c,d,e,f,g,h,i,j,k,l,m,n,o,p,q,r,s,t,u,v,w#}::Indices(position=independent).

   D{#}::PartialDerivative.
   \nabla{#}::Derivative.

   g_{a b}::Metric.
   g^{a b}::InverseMetric.

   \delta{#}::KroneckerDelta.

   R_{a b c d}::RiemannTensor.

   x^{a}::Depends(D{#}).
   x^{a}::Weight(label=num,value=1).

   R_{a b c d}::Depends(\nabla{#}).

   DxaDxb := D_{a}{x^{b}}->\delta^{b}_{a}.

   # can chose lower order approximations by truncating the following pair

   gab := g_{a b} - 1/3 x^{c} x^{d} R_{a c b d}
                  - 1/6 x^{c} x^{d} x^{e} \nabla_{c}{R_{a d b e}}.                    # cdb(gab.000,gab)

   iab := g^{a b} + 1/3 x^{c} x^{d} g^{a e} g^{b f} R_{c e d f}
                  + 1/6 x^{c} x^{d} x^{e} g^{a f} g^{b g} \nabla_{c}{R_{d f e g}}.    # cdb(iab.000,iab)

   gab := g_{a b} -> @(gab).
   iab := g^{a b} -> @(iab).

   gam := 1/2 g^{d c} (D_{a}{g_{c b}} + D_{b}{g_{a c}} - D_{c}{g_{a b}}).             # cdb(gam.001,gam)

   substitute   (gam,gab)
   substitute   (gam,iab)
   distribute   (gam)              # cdb(gam.002,gam)
   unwrap       (gam)              # cdb(gam.003,gam)
   product_rule (gam)              # cdb(gam.004,gam)
   distribute   (gam)              # cdb(gam.005,gam)
   substitute   (gam,DxaDxb)       # cdb(gam.006,gam)
   eliminate_kronecker (gam)       # cdb(gam.007,gam)
   sort_product   (gam)            # cdb(gam.008,gam)
   rename_dummies (gam)            # cdb(gam.009,gam)
   canonicalise   (gam)            # cdb(gam.010,gam)

   def truncate (obj,n):

       ans = Ex(0)  # create a Cadabra object with value zero

       for i in range (0,n+1):
          foo := @(obj).
          bah  = Ex("num = " + str(i))
          distribute  (foo)
          keep_weight (foo, bah)
          ans = ans + foo

       return ans

   gam = truncate (gam,2)   # cdb (gam.101,gam)  # allow up to 2nd order in x^a

   # ==========================================================================
   # the remaining code is just for pretty printing

   {x^{a},g^{a b},R_{a b c d},\nabla_{e}{R_{a b c d}}}::SortOrder.

   def get_term (obj,n):

       foo := @(obj).
       bah  = Ex("num = " + str(n))
       distribute  (foo)
       keep_weight (foo, bah)

       return foo

   def reformat (obj,scale):

      foo  = Ex(str(scale))
      bah := @(foo) @(obj).

      distribute     (bah)
      sort_product   (bah)
      rename_dummies (bah)
      canonicalise   (bah)
      factor_out     (bah,$x^{a?},g^{b? c?}$)
      ans := @(bah) / @(foo).

      return ans

   gam1 = get_term (gam,1)       # cdb (gam1.301,gam1)
   gam2 = get_term (gam,2)       # cdb (gam2.301,gam2)

   gam1 = reformat (gam1,  3)    # cdb (gam1.301,gam1)
   gam2 = reformat (gam2, 12)    # cdb (gam2.301,gam2)

   Gamma  := @(gam1) + @(gam2).  # cdb (Gamma.301,Gamma)
   Scaled := 12 @(Gamma).        # cdb (Scaled.301,Scaled)

\end{cadabra}

\subsection*{The metric connection in Riemann normal coordinates}

\begin{dgroup*}
   \begin{dmath*} \Gamma^{d}_{a b} = \cdb{Gamma.301} \end{dmath*}
\end{dgroup*}

\begin{dgroup*}
   \begin{dmath*} 12 \Gamma^{d}_{a b} = \cdb{Scaled.301} \end{dmath*}
\end{dgroup*}

\vspace{20pt}

\begin{latex}
   \begin{dgroup*}
      \begin{dmath*} \Gamma^{d}_{a b} = \cdb{Gamma.301} \end{dmath*}
   \end{dgroup*}

   \begin{dgroup*}
      \begin{dmath*} 12 \Gamma^{d}_{a b} = \cdb{Scaled.301} \end{dmath*}
   \end{dgroup*}
\end{latex}

% uncomment the following to get more detail of the computations

% \begin{dgroup*}
%    \begin{dmath*} \cdb{gam.002} \end{dmath*}
%    \begin{dmath*} \cdb{gam.003} \end{dmath*}
%    \begin{dmath*} \cdb{gam.004} \end{dmath*}
%    \begin{dmath*} \cdb{gam.005} \end{dmath*}
%    \begin{dmath*} \cdb{gam.006} \end{dmath*}
%    \begin{dmath*} \cdb{gam.007} \end{dmath*}
%    \begin{dmath*} \cdb{gam.008} \end{dmath*}
%    \begin{dmath*} \cdb{gam.009} \end{dmath*}
%    \begin{dmath*} \cdb{gam.010} \end{dmath*}
% \end{dgroup*}

\end{document}
