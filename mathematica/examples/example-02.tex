% based on example 7 in pythontex_gallery
% https://github.com/gpoore/pythontex/

\documentclass[12pt]{mmalatex}
\usepackage{examples}

\begin{document}

\section*{A table of derivatives and anti-derivatives}

This example is based upon a nice example in the Pythontex gallery, see
\ \url{https://github.com/gpoore/pythontex/}.
It uses a tagged block to capture the Mathematica output for later use
in the body of the LaTeX table.

\lstset{numbers=left}

\begin{minipage}[t]{0.65\textwidth}
\begin{mathematica}
   (* Create a list of functions to include in the table *)
   fun = {Sin[x],      Cos[x],      Tan[x],
          ArcSin[x],   ArcCos[x],   ArcTan[x],
          Sinh[x],     Cosh[x],     Tanh[x]};

   eol = {"\\\\" ,     "\\\\",      "\\\\",
          "\\\\[5pt]", "\\\\[5pt]", "\\\\[5pt]",
          "\\\\",      "\\\\",      " "};

   ddxfun = D[#, x] & /@ fun;
   intfun = Integrate[#, x] & /@ fun;

   ddxfunHold = HoldForm[D[#, x]] & /@ fun;
   intfunHold = HoldForm[Integrate[#, x]] & /@ fun;

   (* mmaBeg (CalculusTable) *)
   Do[Print[OutputForm[
      ToString[TeXForm[ddxfunHold[[i]]]] <> "&=" <>
      ToString[TeXForm[ddxfun[[i]]]]     <> "\\quad & \\quad"
      ToString[TeXForm[intfunHold[[i]]]] <> "&=" <>
      ToString[TeXForm[intfun[[i]]]]     <>
      eol[[i]]
      ]], {i,1,9}]
   (* mmaEnd (CalculusTable) *)
\end{mathematica}
\end{minipage}
\hskip 1cm
\begin{minipage}[t]{0.35\textwidth}
\begin{latex}
   \begin{align*}
      \mma {CalculusTable}
   \end{align*}
\end{latex}
\end{minipage}

\clearpage

\begin{align*}
   \mma {CalculusTable}
\end{align*}

\end{document}
