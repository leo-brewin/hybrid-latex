\documentclass[12pt]{pylatex}
\usepackage{examples}

\newbox\SlashInput
\setbox\SlashInput=\hbox{\tt\Large\verb|\Input|}

\begin{document}

\section*{Using {\copy\SlashInput}}

% \vspace{-5pt}

This simple example shows how {\tt\small\verb|\Input|} can be used in include other LaTeX sources within
the host source. Note the use of nested {\tt\small\verb|\Input|}'s. The merged file can be inspected by running

\begin{lstlisting}
   merge-src.py -i example-10.tex -o merged.tex
\end{lstlisting}

from the command line. The merged file will be named {\tt\small merged.tex}.

\PySetup{action=verbatim}

% Note: The * in the 1st column is to ensure the python preprocessor does not respond
%       to any of the \begin{...} or \end{...} lines. The \end{document} line is the
%       real problem as it terminates the preprocessing.
%       The * is hidden from LaTeX by way of the listinings option gobble=2.

\vspace{-5pt}

% ===================================================================
\section*{Source of example-10.tex}

\bgroup
\bgcolour{white}
\latexstyle
\begin{latex}
*   \Input{./example-10/limits/limits.tex}
*   \Input{./example-10/calculus/calculus.tex}
\end{latex}
\egroup

\vspace{10pt}

% ===================================================================
\section*{Source of example-10/limits/limits.tex}

\bgroup
\bgcolour{white}
\latexstyle
\begin{latex}
*   \section*{Limits}
*
*   \begin{python}
\end{latex}
\pythonstyle
\begin{python}
*      from sympy import *
*      a, n, x, dx = symbols('a n x dx')
*      ans = limit(sin(4*x)/x,x,0)                  # py (ans.301,ans)
*      ans = limit(2**x/x,x,oo)                     # py (ans.302,ans)
*      ans = limit(((x+dx)**2 - x**2)/dx, dx,0)     # py (ans.303,ans)
*      ans = limit((4*n + 1)/(3*n - 1),n,oo)        # py (ans.304,ans)
*      ans = limit((1+(a/n))**n,n,oo)               # py (ans.305,ans)
\end{python}
\latexstyle
\begin{latex}
*   \end{python}
*
*   \begin{align*}
*      &\py*{ans.301}\\
*      &\py*{ans.302}\\
*      &\py*{ans.303}\\
*      &\py*{ans.304}\\
*      &\py*{ans.305}
*   \end{align*}
\end{latex}
\egroup

\clearpage

% ===================================================================
\section*{Source of example-10/calculus/calculus.tex}
\begin{latex}
*   \Input{./example-10/calculus/derivs/derivs.tex}
*   \Input{./example-10/calculus/integrals/integrals.tex}
\end{latex}

% ===================================================================
\section*{Source of example-10/calculus/derivs/derivs.tex}

\bgroup
\bgcolour{white}
\latexstyle
\begin{latex}
*   \section*{Differentiation}
*
*   \begin{python}
\end{latex}
\pythonstyle
\begin{python}
*      ans = diff(x*sin(x),x)                                    # py (ans.501,ans)
*      ans = diff(x*sin(x),x).subs(x,pi/4)                       # py (ans.502,ans)
\end{python}
\latexstyle
\begin{latex}
*   \end{python}
*
*   \begin{align*}
*      &\py*{ans.501}\\
*      &\py*{ans.502}
*   \end{align*}
\end{latex}
\egroup

% ===================================================================
\section*{Source of example-10/calculus/integrals/integrals.tex}

\bgroup
\bgcolour{white}
\latexstyle
\begin{latex}
*   \section*{Integration}
*
*   \begin{python}
\end{latex}
\pythonstyle
\begin{python}
*      a, b, x, y = symbols('a b x y')
*      ans = integrate(2*sin(x)**2, (x,a,b))                     # py (ans.503,ans)
*      ans = Integral(2*exp(-x**2), (x,0,oo))                    # py (lhs.504,ans)
*      ans = ans.doit()                                          # py (ans.504,ans)
*      ans = Integral(Integral(x**2 + y**2, (y,0,x)), (x,0,1))   # py (lhs.505,ans)
*      ans = ans.doit()                                          # py (ans.505,ans)
\end{python}
\latexstyle
\begin{latex}
*   \end{python}
*
*   \begin{align*}
*      &\py*{ans.503}\\
*       \py{lhs.504}&=\Py{ans.504}\\
*       \py{lhs.505}&=\Py{ans.505}
*   \end{align*}
\end{latex}
\egroup

\clearpage

\PySetup{action=show}

\Input{./example-10/limits/limits.tex}
\Input{./example-10/calculus/calculus.tex}

\end{document}
